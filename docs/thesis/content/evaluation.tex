\chapter{Evaluation} \label{evaluation}

This chapter evaluates my photorealistic scene reconstruction pipeline. The pipeline is divided into two primary stages: post-processing the dataset and training a Gaussian splat model. For testing the image post-processing script, I created a scene with challenging conditions, including high angular speeds and overexposure, to evaluate the script's ability to filter out unsuitable images. A scene recorded by the robot was used to test the Gaussian splatting and assess the 3D reconstruction quality. 

\section{Image post-processing script}

To assess the effectiveness of the image post-processing script, I created a dataset with simulated challenging conditions by making rapid movements and positioning a flashlight near the camera. These actions mimicked high angular speeds and overexposure, aiming to introduce blurry and overexposed frames into the dataset. Figure~\ref{fig:keyframes_before_process} shows the saved keyframe images before running the script.

\FloatBarrier
\begin{figure}[htbp]
	\centering
	\includegraphics[width=150mm, keepaspectratio]{figures_jpg/keyframes_before_process.jpg}
	\caption{Saved images from mapping by the \textit{keyframe\_saver} node before processing}
	\label{fig:keyframes_before_process}
\end{figure}
\FloatBarrier

After running the script, which detected and removed blurry and overexposed frames, the processed image set can be seen in Figure~\ref{fig:keyframes_after_process}. The script’s output, detailing deletions and renamings, is shown in Figure~\ref{fig:image_process_script_output}. Examples of detected blurry and overexposed images that the script correctly flagged for deletion are displayed in Figure~\ref{fig:blurry_overexposed_example}.

\FloatBarrier
\begin{figure}[htbp]
	\centering
	\includegraphics[width=150mm, keepaspectratio]{figures_jpg/keyframes_after_process.jpg}
	\caption{Remaining images after running the image post-processing script}
	\label{fig:keyframes_after_process}
\end{figure}

\begin{figure}[htbp]
	\centering
	\includegraphics[width=150mm, keepaspectratio]{figures_jpg/process_script_output.jpg}
	\caption{Terminal output of the image post-processing script}
	\label{fig:image_process_script_output}
\end{figure}


\begin{figure}[htbp]
	\centering
	\includegraphics[width=67mm, keepaspectratio]{figures_jpg/example_for_blurry.jpg}\hspace{1cm}
	\includegraphics[width=67mm, keepaspectratio]{figures_jpg/example_for_overexposed.jpg}\\\vspace{5mm}
	\caption{Examples of blurry (left) and overexposed (right) images successfully detected and deleted by the script}
	\label{fig:blurry_overexposed_example}
\end{figure}
\FloatBarrier

I would like to note that I ran the image post-processing script without including the JSON files that define the poses for the keyframes. This ensures that no additional files appear in the folder shown in the figures, making the contents more straightforward and easier to interpret.

\section{Coordinate transformations}

An example output of the node that collects images and keyframe poses is shown in Figure~\ref{fig:mapping_keyframe_folder}. For each image, a corresponding JSON file is generated, containing the position and orientation of the keyframe where the image was captured. An example of such a pose file for \verb|keyframe_0012| is presented here:

\begin{verbatim}
{
    "position": {
        "x": -0.6387644059150104,
        "y": -0.025155113705648685,
        "z": 0.05172834469538916
    },
    "orientation": {
        "x": -0.5780596251044192,
        "y": -0.41060484573413575,
        "z": 0.4070652047425972,
        "w": 0.5758026133768871
    }
}
\end{verbatim}

The images and their associated pose files share the same filename, differing only in their file extensions.

\begin{figure}[htbp]
	\centering
	\includegraphics[width=150mm, keepaspectratio]{figures_jpg/mapping_keyframe_folder.jpg}
	\caption{Keyframe images and saved poses}
	\label{fig:mapping_keyframe_folder}
\end{figure}

Using the collected keyframes and camera intrinsic parameters, I executed my script to prepare the dataset for generating a Gaussian splatting training. The script processes the data, transforms the point cloud, and outputs the necessary files for the pipeline. A snapshot of the script's execution, including the progress updates for the point cloud transformation, is shown in Figure~\ref{fig:create_splatfacto_script_cli}.

\begin{figure}[htbp]
	\centering
	\includegraphics[width=150mm, keepaspectratio]{figures_jpg/create_splatfacto_script_cli.jpg}
	\caption{Output of the script}
	\label{fig:create_splatfacto_script_cli}
\end{figure}

The folder structure generated by the script is depicted in Figure~\ref{fig:Nerfstudio_input_by_my_script}. This structure includes a folder containing the keyframe images, a transformed point cloud file, and a \verb|transforms.json| file. The \verb|transforms.json| file encodes the transformation matrices and camera intrinsics corresponding to each image's pose. A snippet of this file for \verb|keyframe_0012| is shown below:

\begin{verbatim}
    {
            "file_path": "images/keyframe_0012.png",
            "transform_matrix": [
                [
                    0.0022387671847390878,
                    0.999982376839713,
                    0.0054985390324295835,
                    0.05172834469538916
                ],
                [
                    -0.3314031594950285,
                    0.005929748971094606,
                    -0.9434706057709763,
                    0.6387644059150104
                ],
                [
                    -0.9434865837934345,
                    0.0002899778240120221,
                    0.3314105944514393,
                    0.025155113705648685
                ],
                [
                    0.0,
                    0.0,
                    0.0,
                    1.0
                ]
            ],
            "intrinsics": {
                "fl_x": 394.355,
                "fl_y": 393.688,
                "cx": 318.716,
                "cy": 207.653,
                "h": 400,
                "w": 640
            }
        },
\end{verbatim}


\begin{figure}[htbp]
	\centering
	\includegraphics[width=150mm, keepaspectratio]{figures_jpg/Nerfstudio_input_by_my_script.jpg}
	\caption{Created input for Nerfstudio by the script}
	\label{fig:Nerfstudio_input_by_my_script}
\end{figure}

\FloatBarrier
\section{Gaussian Splatting}

To evaluate the capabilities of my pipeline and Gaussian splatting, we conducted an experiment in the Nokia office using the robot. The robot was driven within a rectangular area while the mapping node was active. During the experiment, we also started the keyframe saver and point cloud saver nodes on the laptop to capture images, their corresponding keyframe poses, and the point cloud as the robot moved. After performing image post-processing and running the dataset preparation script, we trained the \verb|splatfacto| model. The results of the reconstruction can be observed in Figure~\ref{fig:nokia_splatfacto_ours_1} and Figure~\ref{fig:nokia_splatfacto_ours_2}.

The reconstructed scene, however, was suboptimal and exhibited critical issues. While some unique features in the environment were recognizable, the overall quality was far from the desired outcome. After investigating the problem, I made the following key observations:

\begin{itemize}
    \item Image Quality: The captured images were of sufficient quality for training the model.
    \item Pose Accuracy: The keyframe poses appeared to be consistent with the observed scene geometry. To evaluate this, my advisor and I manually estimated the distances and rotations between distant keyframes and compared them with the pose data provided by the camera. While this method provided a general sense of agreement, it lacked the precision required for pixel-perfect validation, so the results should be interpreted with caution, this also could cause suboptimal results.
    \item Transformation Validation: The transformations were verified during training by inspecting the keyframe trajectory in the viewer. The trajectory accurately reflected the robot's real-world movements. Additionally, the viewer allowed us to display the point cloud during training, which confirmed proper alignment between the poses and the point cloud. This alignment is evident in Figure~\ref{fig:trajectory_and_pointcloud} and Figure~\ref{fig:pointcloud_debug}, where both the trajectory and point cloud depict the same corner of the environment.
    \item Point Cloud Issues: In my opinion, the primary cause of the suboptimal results was the point cloud itself. As seen in Figure~\ref{fig:pointcloud_debug}, the point cloud contained a significant amount of noise points, these do not exist in the real environment. This noise severely impacted the reconstruction, as Gaussian splatting is highly sensitive to such artifacts. Since Gaussian splatting fits a Gaussian to each point, even noise points contribute to the reconstruction as surfaces. Additionally, there was a bump in the floor, causing certain points of the floor to be slightly tilted, as illustrated in Figure~\ref{fig:pointcloud_debug1}.
\end{itemize}

\begin{figure}[htbp]
	\centering
	\includegraphics[width=150mm, keepaspectratio]{figures_jpg/nokia_splatfacto_ours1.jpg}
	\caption{Reconstructed map in the Nokia office with the \textit{splatfacto} model using poses from the robot VIO (robot's view)}
	\label{fig:nokia_splatfacto_ours_1}
\end{figure}

\begin{figure}[htbp]
	\centering
	\includegraphics[width=150mm, keepaspectratio]{figures_jpg/nokia_splatfacto_ours2.jpg}
	\caption{Reconstructed map in the Nokia office with the \textit{splatfacto} model using poses from the robot VIO (top view)}
	\label{fig:nokia_splatfacto_ours_2}
\end{figure}

\begin{figure}[htbp]
	\centering
	\includegraphics[width=150mm, keepaspectratio]{figures_jpg/trajectory_and_pointcloud_debug.jpg}
	\caption{Point cloud and keyframe trajectory during training}
	\label{fig:trajectory_and_pointcloud}
\end{figure}

\begin{figure}[htbp]
	\centering
	\includegraphics[width=150mm, keepaspectratio]{figures_jpg/pointcloud_debug.jpg}
	\caption{A unique corner in the point cloud}
	\label{fig:pointcloud_debug}
\end{figure}

\begin{figure}[htbp]
	\centering
	\includegraphics[width=150mm, keepaspectratio]{figures_jpg/pointcloud_debug1.jpg}
	\caption{Tilted floor in the point cloud}
	\label{fig:pointcloud_debug1}
\end{figure}


To validate the usability of the reconstruction for mapping scenarios, we compared our results with those produced by COLMAP, it estimated a sparse point cloud of the scene and the poses of the cameras in the scene. The COLMAP-generated reconstruction has significantly higher quality, as can be seen in Figure~\ref{fig:nokia_splatfacto_colmap_1} and Figure~\ref{fig:nokia_splatfacto_colmap_2}. The training process can be observed on a video at Appendix~\ref{yt_gsplat}.

\begin{figure}[htbp]
	\centering
	\includegraphics[width=150mm, keepaspectratio]{figures_jpg/nokia_splatfacto_1.jpg}
	\caption{Reconstructed map in the Nokia office with the \textit{splatfacto} model using poses calculated by COLMAP}
	\label{fig:nokia_splatfacto_colmap_1}
\end{figure}

\begin{figure}[htbp]
	\centering
	\includegraphics[width=150mm, keepaspectratio]{figures_jpg/nokia_splatfacto_2.jpg}
	\caption{Reconstructed map in the Nokia office with the \textit{splatfacto} model using poses calculated by COLMAP}
	\label{fig:nokia_splatfacto_colmap_2}
\end{figure}

In conclusion, while the reconstruction obtained using COLMAP accurately depicts the real environment, our experiments demonstrated that using the VIO-generated poses and points directly for reconstruction is not effective, as it resulted in poor-quality maps. However, VIO proved valuable in selecting keyframes, significantly reducing the number of frames processed by COLMAP and thus providing a considerable speedup.

Based on these findings, the recommended approach for future work is to leverage the strengths of both VIO and COLMAP. Specifically, VIO keyframes should be used as input for COLMAP reconstruction, ensuring that only the most informative frames are processed. Additionally, the coarse poses provided by VIO can serve as initial estimates for COLMAP, which can then refine these poses during its optimization process. This hybrid approach has the potential to combine the efficiency of VIO with the high accuracy of COLMAP, achieving better reconstructions without excessive computational overhead.
