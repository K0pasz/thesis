\pagenumbering{roman}
\setcounter{page}{1}

\selecthungarian

%----------------------------------------------------------------------------
% Abstract in Hungarian
%----------------------------------------------------------------------------
\chapter*{Kivonat}\addcontentsline{toc}{chapter}{Kivonat}

Jelenleg az iparban és a mindennapi életben egyre fontosabb szerepet töltenek be a mobilis robotok. Alkalmazási területeik rendkívül széleskörűek, a raktári környezetekben végzett rakománykezeléstől kezdve az otthoni robotporszívókig, kórházi kiszolgáló robotokig és autonóm járművekig. Különösen fontos a mozgásuk és navigációjuk elősegítése olyan környezetekben, amelyek előre nem ismertek vagy dinamikusan változnak, például ahol emberek vagy más robotok is jelen vannak. Az ilyen környezetek feltérképezése, valamint a navigációs algoritmusok fejlesztése a robotika egyik gyorsan fejlődő, kiemelten kutatott szegmense.

A feltérképezés alapvető eszközei a LiDAR-ok, a különböző mélységi kamerák, valamint más távolságmérő szenzorok, például a radar, az infravörös és az ultrahangos érzékelők. LiDAR-ból létezik 2D-s és 3D-s változat, ugyanakkor az utóbbit ritkán alkalmazzák magas ára miatt. Amennyiben 3D-s feltérképezést szeretnénk használni (amely sokkal több területen alkalmazható), érdemesebb inkább valamilyen mélységi kamerát használnunk. Ez lehet RGB-D vagy sztereó kamera. Ezen kamerák használatával nem csupán pontfelhőt tudunk felépíteni a környezetünkről, hanem akár képesek lehetünk különböző objektumok detektálására is. Ez azért fontos, mert egy dinamikus környezetben, például raktárban vagy közlekedési szituációban, szükség lehet emberek és más mozgó objektumok kikerülésére, illetve követésére.

Ezen diplomamunka megírása során a feladatom megvizsgálni, milyen módon alkalmazhatunk mélységi kamerákat mobilis robotokon környezetfeltérképezésre,  objektumfelismerésre, valamint fotorealisztikus környezeti rekonstrukcióra. A feltérképezés közben, illetve utána,  tájékozódás közben is cél a felismert személyeket kikerülni, illetve követni a robottal. Jelen munkát a Nokia Bell Labs-nél végeztem Budapesten, ahol lehetőségem nyílt a munkámat egy Turtlebot4 roboton tesztelni. A roboton található egy OAK-D Pro kamera, amelynek segítségével elvégeztem a feltérképezést. Célunknak tekintettük továbbá egy neurális rekonstrukció megvalósítását (\textit{NeRF} vagy \textit{Gaussian Splatting} segítségével) a feltérképezett környezetről, hogy fotorealisztikus modellt kapjunk a robotról és a környezetéről.


\vfill
\selectenglish


%----------------------------------------------------------------------------
% Abstract in English
%----------------------------------------------------------------------------
\chapter*{Abstract}\addcontentsline{toc}{chapter}{Abstract}

Currently, mobile robots play an increasingly important role in industry and everyday life. Their applications are highly diverse, ranging from load handling in warehouse environments to household robot vacuums, hospital service robots, and autonomous vehicles. A key challenge is enabling their movement and navigation in environments that are either unknown in advance or dynamically changing, such as those with moving people or other robots. Mapping such environments, along with developing navigation algorithms, is one of the most rapidly developing and heavily researched segments of robotics.

The basic tools for mapping include LiDARs, various depth cameras, and other distance measurement sensors such as radar, infrared, and ultrasonic sensors. LiDARs come in 2D and 3D versions, though the latter is rarely used due to its high cost. If 3D mapping is required (which offers significantly more possibilities), depth cameras are often a more practical choice. These can include RGB-D or stereo cameras. By using such cameras, we can not only construct a point cloud of the surroundings but also detect various objects in the environment. This capability is crucial in dynamic settings, such as warehouses or transportation scenarios, where avoiding or tracking people and other moving objects may be necessary.

My task is to investigate how we can use depth cameras on mobile robots for environmental mapping, object detection, and photorealistic environment reconstruction. During operation, my aim is to avoid/follow the recognized persons with the robot. I wrote this thesis at Nokia Bell Labs in Budapest. With the help of my advisor, dr. Gábor Sörös, I had the opportunity to test my solutions on a real Turtlebot4. The device has an OAK-D Pro camera, which I was able to use for mapping and recognition. We also set as goal to create a neural reconstruction (\textit {NeRF} or \textit{Gaussian Splatting}) of the mapped environment to get a photorealistic picture of the robot and its surroundings.

\vfill
\selectthesislanguage

\newcounter{romanPage}
\setcounter{romanPage}{\value{page}}
\stepcounter{romanPage}