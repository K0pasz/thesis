\chapter{Summary} \label{summary}

The goal of this thesis was to implement an environment mapping system on a mobile robot equipped with a stereo camera at Nokia Bell Labs Budapest. Additionally, the robot was intended to detect and interact with people by either following or avoiding them. Another objective was to establish a pipeline for generating photorealistic reconstructions of the mapped environment.

In the first semester of my Master of Science thesis, I became familiar with the Turtlebot4 mobile robot and its OAK-D stereo camera. I conducted experiments with the camera, the Spectacular AI SDK, photorealistic reconstructions, mapping, and object detection.

The Turtlebot4 offered considerable potential due to its advanced hardware: a Raspberry Pi 4 controller running Ubuntu 22.04 with ROS2 compatibility, and the OAK-D Pro stereo camera. The camera's capabilities included generating depth maps using its stereo lenses and accurately determining object positions using its Visual Processing Unit (VPU). Furthermore, with its Inertial Measurement Unit (IMU), the robot's poses could be calculated in real-time. Leveraging the Spectacular AI SDK, I explored its examples, which demonstrated the SDK's value for mapping and object detection tasks.

Photorealistic scene reconstruction is gaining traction, and as an additional goal, we sought to achieve reconstructions solely using data recorded during mapping. In the first semester, I tested NeRFs and Gaussian Splatting with a handheld OAK-D camera, yielding promising results that motivated further exploration in the second semester.

I initially experimented with RTAB-Map for mapping due to its support for OAK-D cameras. Unfortunately, it proved unreliable, introducing lags that hindered practical use. Interestingly, RTAB-Map's iOS application performed smoothly on my phone. Separately, I implemented a ROS2 node for person detection, which published detected individuals' coordinates. This feature leveraged the camera's VPU to run neural networks for real-time object detection. However, a critical issue arose: once a person was detected, the IMU buffer overflowed, freezing the system. Even Luxonis’ original object detection example code encountered the same problem, indicating a firmware-level issue. Consequently, I could not achieve robust person detection on the robot.

In the second semester, I explored NVIDIA's \verb|nvblox| and \verb|isaac-vslam| for mapping, implemented a custom mapping node using the Spectacular AI SDK, and developed a pipeline for creating photorealistic reconstructions.

Since RTAB-Map was unsuitable, I turned to \verb|nvblox|, a GPU-accelerated tool for voxel-based mapping compatible with the robot's camera. However, hardware limitations prevented full functionality; my GTX 1660 Ti GPU lacked the required 8 GB of VRAM, offering only 6 GB. I then explored \verb|isaac-vslam|, another NVIDIA tool for Visual SLAM that uses GPU acceleration. Its simulations can run in NVIDIA Omniverse, which supports creating diverse environments. Unfortunately, Omniverse requires an RTX GPU for hardware-accelerated ray tracing, which my GTX GPU could not provide. Ultimately, deploying \verb|nvblox| or \verb|isaac-vslam| on the robot would require a Jetson with at least 8 GB of VRAM, but budget constraints precluded this option.

In the final phase of my thesis, I developed a ROS2 mapping node using the Spectacular AI SDK to compute the robot's poses and publish its environment as a point cloud. Additional nodes, running on a notebook, collected poses, images, and point clouds at keyframes. After mapping, my post-processing script removed blurry and overexposed images and their associated poses. Another script prepared the input data for Gaussian splatting by transforming robot poses and the point cloud into the required coordinate system and embedding camera intrinsics into a JSON file describing keyframe transformations. Using this dataset, I trained a Gaussian splatting model and generated splats. However, due to noise and misalignments in the point cloud, the results did not match our expectations for photorealistic reconstruction. To prove that this approach could be used well in environment mapping I used COLMAP to predict the poses of the keyframes and a point cloud merely from images. With this approach I could achieve realistic results from the environment.

In the future, if the camera's manufacturer could solve the issues experienced in object detection, it would be possible to add a person detection feature to my mapping node. Another improvement would be a filtering and adjusting feature for the point cloud with which we would be able to create lifelike scenes from the mapped environment.
